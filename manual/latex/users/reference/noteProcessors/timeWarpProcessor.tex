\section{Time Warp Processor}\label{timewarpProcessor}

parameters: TimeWarpString

Warps time in the same way as Csound t-statement, but does not require
"t" to be used. Statements are in alternating pairs of beat number and
tempo.

From the Csound Manual:

Time and Tempo-for-that-time are given as ordered couples that define
points on a "tempo vs. time" graph. (The time-axis here is in beats so
is not necessarily linear.) The beat-rate of a Section can be thought of
as a movement from point to point on that graph: motion between two
points of equal height signifies constant tempo, while motion between
two points of unequal height will cause an accelarando or ritardando
accordingly. The graph can contain discontinuities: two points given
equal times but different tempi will cause an immediate tempo change.

Motion between different tempos over non-zero time is inverse linear.
That is, an accelerando between two tempos M1 and M2 proceeds by linear
interpolation of the single-beat durations from 60/M1 to 60/M2.

The first tempo given must be for beat 0.

\begin{itemize}
\item
  Beat values for beat/tempo pairs should related to the score *before*
  any time behavior is applied. For example, for the following score:

\begin{verbatim}
i1 0 1 2 3 4
i1 1 1 3 4 5 
i1 2 1 3 4 5 
i1 3 1 3 4 5 
\end{verbatim}

  if it is in a GenericScore SoundObject of duration 20, if you want the
  tempo to decrease in half by the last note, you would enter a value
  for the processor as "0 60 3 30" and not "0 60 20 30"
\item
  If you're using a time behavior of "Repeat", remember that time
  behavior is applied *after* noteProcessors, and the resulting score
  will be a time warped score repeated x times and *NOT* a score
  repeated x time and then timewarped
\item
  Time Warping, when used with a time behavior of "Scale", be aware that
  estimating the final tempo of the object may be tricky, as the scaling
  will alter the duration of notes.
\end{itemize}
