\section{SoundObject Library}\label{soundObjectLibrary}

\subsection{Introduction}

The SoundObject Library is a place where one can store soundObjects from
the timeline. The library can simply be a place to store a soundObject
idea that the user may want to user later, but more importantly it
allows for Instances of the soundObject to be made. Instances of
soundObjects point to and soundObject in the library and when the
instance is generating its score, it will actually call the soundObject
in the library to generate its score and then apply it's own properties
and noteProcessors to the generated score. Updating a soundObject in the
library will then update all instances of that soundObject. This feature
is useful to represent the idea of a motive, with instances of the
motive allowing to have transformations by use of noteProcessors.

\subsection{Usage}

The general pattern of usage for the SoundObject Library entails:

\begin{enumerate}
\def\labelenumi{\arabic{enumi}.}
\item
  Add SoundObject to the Library. This is done by selecting a
  soundObject, right-clicking the soundObject to open up the popup menu
  and selecting "Add to SoundObject Library".
\item
  After doing this, your soundObject will have been added to the library
  and the soundObject on the timeline will have been replaced with an
  Instance soundObject which will be pointing to the soundObject now in
  the library.
\item
  At this point, the user can now take advantage of the library by
  making copies of the instance object on the timeline and pasting in
  more instances. These instances can be placed anywhere, have different
  durations and time behaviors, as well as have their own individual
  noteProcessors. This allows expressing ideas such as "This is an
  instance of the primary motive (soundObject in the library) but
  transposed up a major 3rd, retrograded, and inverted", or an idea like
  "I've got drum pattern A in the library and I have instances of it
  here and here and ...".

  \begin{quote}
  \textbf{Note}

  When copying and pasting Instance soundObjects, they are all pointing
  to the soundObject in the library.
  \end{quote}
\item
  You can also then make instances of soundObjects in the library by
  opening up the SoundObject Library dialog (available from the Window
  menu or by using the F4 shortcut key). There you have the following
  options:

  \begin{description}
  \item[Copy]
  This makes a copy of the selected SoundObject and puts it in the
  buffer. This is a copy of the original soundObject and not an
  Instance. After copying to the buffer, you can paste as normal on the
  timeline.
  \item[Copy Instance]
  This makes a Instance of the selected SoundObject and puts it in the
  buffer. This Instance will point to the original soundObject. After
  copying to the buffer, you can paste as normal on the timeline.
  \item[Remove]
  This will remove the selected SoundObject from the library.
  \end{description}
\item
  You can also then edit the soundObject in the library from within the
  SoundObject Library dialog by selecting the soundObject in the list.
  The editor for the SoundObject will appear below.

  \begin{quote}
  \textbf{Note}

  Editing the SoundObject in the library will affect all instances of
  that SoundObject.
  \end{quote}
\end{enumerate}
