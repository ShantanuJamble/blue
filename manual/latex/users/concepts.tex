\chapter{Concepts}\label{concepts}

\section{Rendering}\label{rendering}

\subsection{Introduction}

blue offers a number of ways to render a project, each method useful for
different purposes you may have. The following section explains the ways
which blue is able to render a .blue project file as well as possible
use-case scenarios to guide you on how that feature may be useful.

\subsection{Render Methods}

Rendering in uses the project's Real-Time render settings to render the
project in realtime to the users soundcard. This is the most common
rendering one will likely use. To use this, you can press the play
button at the top of the window, use the
"Project-\textgreater{}Render/Stop Project" menu option, or use the F9
shortcut key.

blue's general method of communicating with Csound is by generating a
CSD file and calling Csound to run it. For debugging purposes, it is
often useful to see if what you think you are doing inside blue is
matching the generated CSD file. You can generate the CSD to screen by
using the "Project-\textgreater{}Generate CSD to Screen" menu option, or
by using the ctrl-shift-g shortcut.

If you'd like to generate a CSD from the current project to a file, you
can do so by using the "Project-\textgreater{}Generate CSD to File" menu
option, or by using the ctrl-g shortcut.

The Render to Disk option will use the project's Disk Render Options
from its project properties to generate a soundfile on disk (format of
file depends on how user configures commandline options). I've found
this useful for rendering a wav to disk and converting to MP3 to listen
to on the go so I can review my current work. To render to disk, you can
do so by using the "File-\textgreater{}Render to Disk" menu option or
use the ctrl-shift-F9 shortcut.

The Render to Disk and play option will use the project's Disk Render
Options from its project properties to generate a soundfile on disk and
after finishing, play the generated file using blue's soundfile player,
located in the SoundFile manager tab. This feature is useful if you're
working on a piece that has processing demands beyond what your computer
is capable of in realtime, so you can have it render to disk first and
then play, which will give very smooth playback. It is also useful if
you render in realtime with lower quality settings but use higher
quality settings for final project output, as you can occasionally test
what the output would be like using this feature. To use this option,
you can use the "File-\textgreater{}Render to Disk and Play" menu option
or use the ctrl-F9 shortcut.

The Render to Disk and Open option will use the project's Disk Render
Options from its project properties to generate a soundfile on disk and
after finishing, open the generated file using the configured command
with the program options. This command may differ from the external
command to use with "Render to Disk and Play". To use this option, you
can use the "File-\textgreater{}Render to Disk and Open" menu option.

%%
\section{SoundObjects}\label{conceptsSoundObjects}

\subsection{Introduction}

The concept of SoundObjects is the foundation of blue's design in
organizing musical ideas. This section will discuss what is a
SoundObject, how this idea is implemented in blue, and strategies on how
to use the concept of SoundObjects in organizing your own musical work.

\subsection{What is a SoundObject?}

SoundObjects in blue represent a \emph{perceived sound idea, whether it
be a single atomic sound event or aggregate of other sound objects}. A
SoundObject on the timeline can represent many things, whether it is a
single sound, a melody, a rhythm, a phrase, a section involving phrases
and multiple lines, a gesture, or anything else that is a perceived
sound idea.

Just as there are many ways to think about music, each with their own
model for describing sound and vocabulary for explaining music, there
are a number of different SoundObjects in blue. Each SoundObject in blue
is useful for different purposes, with some being more appropriate for
expressing certain musical ideas than others. For example, using a
scripting object like the PythonObject or RhinoObject would service a
user who is trying to express a musical idea that may require an
algorithmic basis, while the PianoRoll would be useful for those
interested in notating melodic and harmonic ideas. The variety of
different SoundObjects allows for users to choose what tool will be the
most appropriate to express their musical ideas.

Since there are many ways to express musical ideas, to fully allow the
range of expression that Csound offers, blue's SoundObjects are capable
of generating different things that Csound will use. Although most often
they are used mostly for generating Csound SCO text, SoundObjects may
also generate ftables, instruments, user-defined opcodes, and everything
else that would be needed to express a musical idea in Csound.

Beyond each SoundObject's unique capabilities, SoundObjects do share
common qualities: name, start time, duration, end time. Most will also
support a Time Behavior (Scale, Repeat, or None) which affects how the
notes generated by the SoundObject will be adjusted-\/-if at all-\/-to
the duration of the SoundObject. Most will also support NoteProcessors,
another key tool in Csound for manipulating notes generated from a
SoundObject. All SoundObjects also support a background color property,
used strictly for visual purposes on the timeline.

%%
\section{}\label{conceptsPolyObjects}

\subsection{Introduction}

PolyObjects are, in my opinion, one of the most powerful tools in blue.
They provide encapsulation of any grouping of SoundObjects, any way you
like, into one logical SoundObject. Once encapsulated, the rest of
blue's powerful features (namely NoteProcessors) can be leveraged
against the PolyObject, and that is when one really begins to realize
the benefits of using PolyObjects.

\subsection{Basic}

To repeat what is stated in the reference documentation, a PolyObject
can be seen as a container of other SoundObjects. It contains its own
series of SoundLayers, just like the main score, and, like the main
score, one can add as many SoundLayers and SoundObjects as one likes to
a PolyObject.

Think of the main score as a big box that holds sounds. Now, we can put
whatever sounds we want in the box, but we can also put smaller boxes
inside the main box; these smaller boxes are our PolyObjects. We can put
sounds inside the smaller boxes, just like we can in the big box, and we
can arrange them in the same manner too. When we put sound A in the
upper-right corner of a smaller box, it will stay in that corner no
matter where we move our box inside of the bigger box. However, it is
the position and arrangement of the sounds themselves relative to the
main score that is important and is, ultimately, the reason to use
PolyObjects. With sound A in the upper-right corner of our small box,
its relative position in the main box will be dependent on the smaller
box's position in the big box. Keep in mind, too, that we can put boxes
inside of boxes inside of boxes, as many layers as we like. And when you
realize that you can change the sounds inside of any box by applying a
NoteProcessor to it (like making the box metal instead of cardboard),
you begin to see the power of using PolyObjects.

The way it works is, you create a PolyObject in a SoundLayer like you
would any other SoundObject; by right-clicking on the SoundLayer and
selecting "Add New PolyObject;" it should be at the very top. After
doing this, the PolyObject is empty and will not generate any notes in
its current state. You have to double-click on it in order to edit the
contents of the PolyObject. Once you're editing the empty PolyObject,
the score editor looks a lot like it does when you're editing an empty
blue score. That's because the main score is just one big container for
notes, like PolyObjects (in fact, in the code, the main score IS a
PolyObject).

You can see which container you are currently editing by looking at the
buttons just underneath the tabs for "score," "orchestra," etc. When
you're editing the main score, there will only be one button there
called "root;" you're editing the root container. When you're editing a
PolyObject called "Group A," you'll see two buttons; "root" and "Group
A." Pressing any of these buttons will change the view to edit that
container.

Once a PolyObject has been created, you can add and arrange SoundObject
on it just like you do in the main score. Each PolyObject is going to
have its own Snap and Time Display settings too, so you may want to set
those accordingly. Add SoundLayers just as you would in the main score
too.

After you've added all of the notes you wish to the PolyObject, click on
the "root" button and feel free to move the PolyObject around in the
main score, just as you would any other SoundObject. Add NoteProcessors
too, or alter the Time Behavior; these changes, in turn, will be applied
to the contents of the PolyObject.

\subsection{Advanced}

Before tackling some of the more advanced concepts of PolyObjects, a
couple of clarifications must be made. When this document refers to
editing the actual PolyObject, that means editing it as a SoundObject
(moving it around on the timeline, stretching it, editing its
properties, etc.). This is very different from editing the contents of
said PolyObject. The distinction will be made in this section whenever
applicable.

\subsubsection{Time}

PolyObjects can have their Time Behavior changed like any other
SoundObject when one is editing the actual PolyObject. This is very
important to keep in mind, because your sounds within the PolyObject may
span 10 beats, but if you leave the Time Behavior settings for the
PolyObject to the default (Scale), they may in actuality span only two
beats, depending on the size of your PolyObject. One can test this with
the Test button: shrink a PolyObject that has the Scale Time Behavior
and the notes within it will shrink as well.

There are a couple of easy ways to make sure that the sounds you place
in a PolyObject remain to scale in the parent container, or the root
score. One method is to right-click on the PolyObject itself and click
"Set Subjective Time to Objective Time." This will change the length of
the PolyObject to be equal to the beat at the end of the last
SoundObject within it. In other words, it will expand or contract to the
actual total length of all of the notes within it. The other method is
to change the Time Behavior of the PolyObject to None. Blue will not do
any processing of the length of the notes within the PolyObject; they'll
just be played exactly the way they are in the PolyObject.

There are advantages and disadvantages to each method. The first is good
because it allows one to see the actual length of the PolyObject within
the main score, but you have to remember to click "Set Subjective Time
to Objective Time" each and every time you make a change to the
PolyObject's contents. If you set the Time Behavior to None, the length
of the PolyObject may appear to be two beats, but it may actually be ten
or 20 or whatever; you don't know until you edit the contents of that
PolyObject. However, with a Time Behavior of None, you can edit the
contents of the PolyObject frequently and drastically without having to
worry if your changes will be accurately represented in the CSD
rendering. A good way to work may be to set a new PolyObject's Time
Behavior to None and edit the contents of it as much as you like. Then,
when it sounds satisfactory, you can click "Set Subjective Time to
Objective Time" on it, so that you may edit the rest of the score while
having an accurate picture of the actual length of the PolyObject. Keep
in mind that you may use the "Set Subjective Time to Objective Time"
feature on any SoundObject with any Time Behavior setting.

One can lay down a series of sounds in a PolyObject and setup the
PolyObject to loop them. The Time Behavior of the PolyObject itself
needs to be set to Repeat, and the notes will loop for as long as the
PolyObject's length. When the PolyObject's contents actually start
looping in relationship to each other is determined by the Repeat Point;
one can edit the Repeat settings in the property box of any SoundObject.
The Repeat Point specifies the beat at which the group of notes will
begin after the previous note group begins. If the Use Repeat Point is
not checked, the next note group will begin immediately after the
previous note group's last note ends.

An example would be a PolyObject whose contents are three one-beat notes
played back-to-back:

\begin{verbatim}
i1 0.0 1.0 3 4 5
i2 1.0 1.0 3 4 5
i3 2.0 1.0 3 4 5
\end{verbatim}

If one set the Time Behavior to Repeat without using a Repeat Point and
stretched the length of the actual PolyObject to six beats, six one-beat
notes would play back-to-back, like this:

\begin{verbatim}
i1 0.0 1.0 3 4 5
i2 1.0 1.0 3 4 5
i3 2.0 1.0 3 4 5
i1 3.0 1.0 3 4 5
i2 4.0 1.0 3 4 5
i3 5.0 1.0 3 4 5
\end{verbatim}

However, if one were to change the PolyObject to use Repeat Points and
set the repeat point to 2.5, this would be the result:

\begin{verbatim}
i1 0.0 1.0 3 4 5
i2 1.0 1.0 3 4 5
i3 2.0 1.0 3 4 5
i1 2.5 1.0 3 4 5
ji2 3.5 1.0 3 4 5
i3 4.5 1.0 3 4 5
i1 5.0 1.0 3 4 5
\end{verbatim}

Note that there are more notes being played this time. Why? Because with
a Time Behavior of Repeat, blue will try to fit as many notes that it
can into the PolyObject using the repeat settings given, without
violating the order and length of the PolyObject's note arrangement.
With the previous example, blue played the second group of three notes
at beat 2.5, and realized it could squeeze in a partial third loop. This
third loop begins at beat five, since that is equivalent to beat 2.5 in
the second loop, which is our Repeat Point. It inserted the first note
there and realized it couldn't fit any more notes, since the next note
would begin at beat six and play until beat seven; that's longer than
the length of our PolyObject, which is six.

\subsubsection{NoteProcessors and PolyObjects}

NoteProcessors are great when used in conjunction with PolyObjects. They
can be very powerful, but you have to be careful of a few gotcha's.

One thing to be aware of with PolyObjects and NoteProcessors are
p-fields. The user must make sure that the p-field(s) that is being
operated on by the Note Processor(s) is the same for all sounds within
the PolyObject. If one has a lot of typical notes in a PolyObject where
p4 equals the pitch, and one errant note where p4 equals phase, for
instance, things could get disastrous if one tries to apply a
PchAddProcessor to the whole PolyObject. Please refer to the Best
Practices section for more information.

Everything that is to follow in this section can be summed up in one
important fact: NoteProcessors get applied to SoundObjects before the
Time Behavior does. What this means is, if you have a PolyObject with
Repeat Time Behavior and set it up to loop twice, and have a
LineAddProcessor on that same PolyObject, you will have a group of notes
looped twice with the LineAddProcessor behavior applied twice; that is,
applied individually to each looped group of notes.

So, to borrow from the example in Repeat/Looping Time Behavior, our
group of three notes with no NoteProcessor and no Repeat Point would
render like this:

\begin{verbatim}
i1 0.0 1.0 3 4 5
i2 1.0 1.0 3 4 5
i3 2.0 1.0 3 4 5
i1 3.0 1.0 3 4 5
i2 4.0 1.0 3 4 5
i3 5.0 1.0 3 4 5
\end{verbatim}

If I add a LineAddProcessor to p-field four, starting from zero and
going to two, here's what the result would look like:

\begin{verbatim}
i1 0.0 1.0 3.0 4 5
i2 1.0 1.0 4.0 4 5
i3 2.0 1.0 5.0 4 5
i1 3.0 1.0 3.0 4 5
i2 4.0 1.0 4.0 4 5
i3 5.0 1.0 5.0 4 5
\end{verbatim}

This may be the desired effect, and it may not. In order to apply a
NoteProcessor after the Time Behavior has been applied, take the
following steps:

\begin{enumerate}
\def\labelenumi{\arabic{enumi}.}
\item
  Apply the Time Behavior to you PolyObject, with no NoteProcessor on
  it.
\item
  Right-click on the PolyObject and select "Convert to PolyObject." What
  this will do is embed your PolyObject in yet another PolyObject.
\item
  Edit the Properties of the new PolyObject and apply the NoteProcessor
  to it.
\end{enumerate}

If one applies the previous steps to the example cited above, the
rendered notes look like this:

\begin{verbatim}
i1 0.0 1.0 3.0 4 5
i2 1.0 1.0 3.4 4 5
i3 2.0 1.0 3.8 4 5
i1 3.0 1.0 4.2 4 5
i2 4.0 1.0 4.6 4 5
i3 5.0 1.0 5.0 4 5
\end{verbatim}

One thing the user will have to keep in mind, though; when referring to
beats, those beats will be applicable to the notes as they are after the
Time Behavior change. So, in the previous example, in order to get those
results, one has to change the ending beat in the LineAddProcessor to
five instead of two, because five is the beat at which the last note
plays in the looped sequence.

As mentioned previously, PolyObjects can contain other PolyObjects. The
thing to remember when doing this is that each PolyObject's settings,
NoteProcessor's, etc. are local to itself. For blue, the notes generated
from a PolyObject SoundObject are the same as the ones generated from a
GenericScore or any other type of SoundObject, and can be manipulated in
the same way.

\subsection{Best Practices}

Let's say you're making a PolyObject with lots of regular notes whose p4
field represents the pitch. At some point you also want to insert a note
that affects something a bit more irregular, like the phase of the other
notes, and its p4 field represents, say, the amount of variance in the
phase. It would be prudent to separate that irregular note away from the
notes in this PolyObject in order to keep your expectations about the
p-field of the PolyObject as a whole consistent. Without the irregular
note, you can safely say that p4 of PolyObject X (or whatever its
called) represents the pitch of all of the notes it contains. However,
with that extra note in it, p4 has no consistent meaning in PolyObject
X, and you can't safely apply NoteProcessors to it that affect that
p-field.

Co-reliant notes, in this case, would be notes that rely on each other
to make their sound. For instance, if you had a group of notes that fed
their output to a zak channel, then another note to process and play
that channel's sound, those notes would be co-reliant. The reason to do
this is to try to make each SoundObject in the main score independent so
that one can move, remove, and edit each of them without worrying about
affecting other notes. The PolyObjects, effectively, begin to represent
one atomic sound in your score, making it easier to experiment with.

Doing this also allows one to reliably freeze PolyObjects. If one tried
to freeze a PolyObject that did nothing but write its output to a zak
channel, the resulting sound file would be dead silence, and the score
would never work (since a frozen sound does nothing but read and play a
sound file).

That's still possible with co-reliant but dissimilar notes. What one
would do, to use the example in the previous section, is make a
PolyObject that contained the instrument processing the zak channel
output and another embedded PolyObject. The embedded PolyObject would
then contain all of the actual notes writing to the zak channel. Now
what you can do is apply all kinds of NoteProcessors to the embedded
PolyObject, then freeze or do whatever to the parent PolyObject in a
reliable fashion.

It's much easier on the user as a composer to arrange the notes this
way. This enables one to see opportunites for reuse of certain musical
phrases, and implement that repetition easily by adding said PolyObjects
to the SoundObject Library and copying instances wherever they're
needed.

This only really applies if the Time Behavior being used is None,
otherwise a PolyObject's length is always going to represent accurately
its length in the composition. Remember to use "Set Subjective Time to
Objective Time" as needed.

\subsection{Debugging}

When things don't work out the way you expected them too (and they
won't), a great tool to use to help track down the problem is the "Test"
button. This button renders the i-statements that any given SoundObject
will generate, and it comes in very useful with PolyObjects. So, if
you've got a PolyObject that won't render when you Test it, drill down
into it and test its constituent notes, Testing and drilling down into
its sub-PolyObjects as well. You'll eventually find the cause. Examine
NoteProcessors and the Time Behaviors of each SoundObject; these can be
common culprits with sound problems too.

If one follows the best practice "Group co-reliant notes together into
PolyObjects," this also makes it easier to debug problems. You can
isolate PolyObjects in the main score easily by soloing that SoundLayer,
etc. (since they won't rely on other SoundObjects for their sound) and
investigate the problem that way; many times, you'll find that the
problem stems from the PolyObject not being as independent as you'd
thought.

%%
\section{NoteProcessors}\label{conceptsNoteProcessors}

NoteProcessors are a powerful tool for manipulating notes. Each type of
NoteProcessor serves a different purpose to adjust values for pfields.
They can be used to apply musical effects to a group of notes; some
possible uses include humanizing rhythm, added crescendo/decrescendo,
reassigning notes to different instruments, applying time curves, and
transposition of material.

NoteProcessors can be applied on a per-SoundObject basis. They can also
be applied to an entire Layer to affect all SoundObjects in that layer,
as well as applied to a LayerGroup or to the entire Score.

You can add NoteProcessors to SoundObjects by using the SoundObject
Properties window. You can add them to Layers the support them by using
the "N" button the layer's header panel. To apply them to a LayerGroup
or Score, right click the Root Score in the Score Bar and use the popup
menus to choose which item you'd like to apply NoteProcessors to.

NoteProcessors are applied after the notes of a SoundObject, Layer,
LayerGroup, or Score are generated and before time behavior is applied.
Processing starts with the first NoteProcessor in the chain and the
results of that are passed down the chain.

%%
\section{SoundObject Library}\label{soundObjectLibrary}

\subsection{Introduction}

The SoundObject Library is a place where one can store soundObjects from
the timeline. The library can simply be a place to store a soundObject
idea that the user may want to user later, but more importantly it
allows for Instances of the soundObject to be made. Instances of
soundObjects point to and soundObject in the library and when the
instance is generating its score, it will actually call the soundObject
in the library to generate its score and then apply it's own properties
and noteProcessors to the generated score. Updating a soundObject in the
library will then update all instances of that soundObject. This feature
is useful to represent the idea of a motive, with instances of the
motive allowing to have transformations by use of noteProcessors.

\subsection{Usage}

The general pattern of usage for the SoundObject Library entails:

\begin{enumerate}
\def\labelenumi{\arabic{enumi}.}
\item
  Add SoundObject to the Library. This is done by selecting a
  soundObject, right-clicking the soundObject to open up the popup menu
  and selecting "Add to SoundObject Library".
\item
  After doing this, your soundObject will have been added to the library
  and the soundObject on the timeline will have been replaced with an
  Instance soundObject which will be pointing to the soundObject now in
  the library.
\item
  At this point, the user can now take advantage of the library by
  making copies of the instance object on the timeline and pasting in
  more instances. These instances can be placed anywhere, have different
  durations and time behaviors, as well as have their own individual
  noteProcessors. This allows expressing ideas such as "This is an
  instance of the primary motive (soundObject in the library) but
  transposed up a major 3rd, retrograded, and inverted", or an idea like
  "I've got drum pattern A in the library and I have instances of it
  here and here and ...".

  \begin{quote}
  \textbf{Note}

  When copying and pasting Instance soundObjects, they are all pointing
  to the soundObject in the library.
  \end{quote}
\item
  You can also then make instances of soundObjects in the library by
  opening up the SoundObject Library dialog (available from the Window
  menu or by using the F4 shortcut key). There you have the following
  options:

  \begin{description}
  \item[Copy]
  This makes a copy of the selected SoundObject and puts it in the
  buffer. This is a copy of the original soundObject and not an
  Instance. After copying to the buffer, you can paste as normal on the
  timeline.
  \item[Copy Instance]
  This makes a Instance of the selected SoundObject and puts it in the
  buffer. This Instance will point to the original soundObject. After
  copying to the buffer, you can paste as normal on the timeline.
  \item[Remove]
  This will remove the selected SoundObject from the library.
  \end{description}
\item
  You can also then edit the soundObject in the library from within the
  SoundObject Library dialog by selecting the soundObject in the list.
  The editor for the SoundObject will appear below.

  \begin{quote}
  \textbf{Note}

  Editing the SoundObject in the library will affect all instances of
  that SoundObject.
  \end{quote}
\end{enumerate}


%%
\section{Parameter Automation}\label{parameterAutomation}

\subsection{Introduction}

Blue allows for a few ways to enter in values to automate parameters
over time. The parameters that can be automated are those in
BlueSynthBuilder Instruments, Mixer Effects, Mixer Sends, and Mixer
Volumes. The following sections will describe how to use Parameter
Automation and technical details in how this all works within the
context of blue and Csound.

\subsection{Available Parameters}

Besides the Volume sliders in the mixer and level amount on mixer sends,
the following GUI Parameters are automatable when found in
BlueSynthBuilder instruments and Mixer Effects:

\begin{itemize}
\item
  Knob
\item
  HSlider
\item
  HSliderBank
\item
  VSlider
\item
  VSliderBank
\item
  XYController
\item
  Dropdown
\item
  Checkbox
\end{itemize}

\subsection{Assigning Parameters}

Automations for Parameters are assignable to SoundObject LayerGroup
Layers that exist in the root Score by selecting from the popup menu
that appears when selecting the "A" button on a SoundLayer's panel (on
the left of the timeline). The menu shows what Parameters are available
to use from Instruments and the Mixer. Those with normal color text are
Parameters which are not currently automated, those in green are ones
currently automated in that SoundLayer, and those in orange are ones
which are currently automated on a different SoundLayer. Selecting
normal color text will enable automation for that parameter on the
SoundLayer, while selecting a green colored text one will disable
automation for that parameter, and selecting an orange one will move it
from the layer it is currently on to the one being worked with.

\subsection{Score Timeline Modes}

The Score timeline has three different modes:

\begin{description}
\item[Score]
This is the primary mode for adding and editing SoundObjects. Parameter
Automations will be drawn but are not editable in this mode.
\item[Single Line]
This is the primary mode for editing line points for Parameter
Automations. The user is able to edit a single line at a time per
SoundLayer, and can enter and remove points, modify points by dragging
them, as well as selecting a region and move the the points by dragging
the region. While dragging the region, the region will show that data
will overwrite the area dragged to. This overwriting will not be
committed however until the user releases the mouse button.
\item[Multi Line]
This is the mode for selecting and moving line points for all Parameter
Automations on a single or multiple SoundLayers. The use can click and
drag to create a selection region, then click within the selection
region to move all points from all lines within the region. While
dragging the region, the region will show that data will overwrite the
area dragged to. This overwriting will not be committed however until
the user releases the mouse button.
\end{description}

\subsection{Editing Automations}

To edit line values in a Parameter Automation, first switch to Single
Line mode. Then, for the desired SoundLayer, select which of the
assigned parameters to edit. This can be done either by using the
SoundLayer Edit Panel's Parameter selector (found on the second row
below the row with mute, solo, etc.; only shown when SoundLayer size is
greater than 1), or by right-clicking the SoundLayer in the main area
and selecting from the popup menu (the actively editable Parameter
Automation will be disabled from the popup menu).

Once a parameter is selected, the user can change the color of the line
by using the color box next to the parameter selector panel on the
SoundLayer Edit Panel. To add new points, use the mouse and mouse over
to where you would like to add a point and press the left mouse button.
After pressing, the point will be added, and the user can drag to modify
the just entered point. Releasing the button will complete the addition
of the line point. To edit an existing point, mouse over a point until
it is highlighted in red, then press the left mouse button, drag, and
release when finished. To remove an existing point, mouse over a point
until it is highlighted and press the right mouse button.

For finer control over the line points, right-click on the parameter
line panel when not over an existing line point. A popup menu will
appear and select "Edit Line Points". A dialog showing a table of line
points with time values in one column and parameter values in a second
column. The user can then enter in values by text entry.

\subsection{Technical Details}

blue's parameter automation system is implemented in Csound code in a
few different areas. For instruments and effects, when an Automatable
Parameter is not set for automation, it will replace its replacement key
in its ORC code with a constant value from the UI item. If the parameter
is set to allow automation, then the replacement key is replaced with a
blue generated global k-rate signal name. Therefore, when coding
instruments and effects, the user should be careful to make sure that if
they want the parameter to be automatable that the replacement key is
placed in the code where a k-rate signal is legal to use. If the user is
not expecting to automate that value, then it would be safe to place
that replacement key wherever a constant is allowed. This is also useful
to know when migrating older BSB Instruments and Effects to be used with
the purpose of Automation.

After setting up the Instrument and Effects, blue will then handle
compiling the automations into a way that works with Csound. Currently,
blue will first calculate init statements from where the render start
time is and add that to the CSD to make sure that the global krate
signals will be initiated by the time any instrument could use them.
Then, blue creates instruments for each parameter that will be
automated. The instruments are driven by score and either an instrument
for continuous data or resolution dependent data will be created. Score
is then generated for the parameter automation data that together with
the instruments generated will create the signals.

Care was taken to optimize the generated score and instruments.
Unnecessary score will not be generated if multiple points with the same
value are found as well as when there is a discontinous break in the
values. Values from the parameter automation will also correctly be
generated from render start time until render end time (if used). For
parameters with resolution, every value is a discrete jump, so the
instrument for resolution based parameters will on it's i-time run will
simply set the new value and then turn the instrument instance off to
save CPU cycles.

%%
\section{Command Blocks}\label{commandBlocks}

\subsection{Introduction}

Command Blocks are Csound ORC code with special directions to blue on
how to processes them. The are available to use only within the Global
Orchestra areas of Instruments (not to be confused with the global
orchestra area in the globals manager tab) and were created to help
users build fully-encapsulated instruments, such that an instrument and
all of its necessary parts could be grouped together.

\begin{quote}
\textbf{Note}

This feature has largely been made unnecessary due to new Csound
programming practices that have developed since this was initially
introduced. This feature is left in blue though to support legacy
projects.
\end{quote}

\subsection{Basic Usage}

To use a command block, you will need to wrap the section of orc code
that you want blue to process with the command in the following manner:

\begin{verbatim}
;[command to use]{ 
...Csound Orchestra Code... 
;}
\end{verbatim}

\subsection{Available Commands}

Once blocks are useful to limit how many times a block of code will be
generated in the final CSD and is more for efficiency than anything else
(so Csound doesn't define the same ftable more than it has to, for
example). An example of this:

\begin{verbatim}
;[once]{ 
gaLeft init 0 gaRight init 0 
;}
\end{verbatim}

In this case, if the above gaLeft and gaRight init statement, wrapped in
a once block, is used in multiple instruments, then regardless it will
be generated only once in the final CSD. So a use case for the above
would be if you have all of your instruments output to the above global
variables for extra processing by an always-on reverb, you would want to
add the above to all of the instruments and also the reverb. Then, if
you start a new project and may want to reuse a few of your instruments
as well as the reverb, you can just copy them into your new project's
orchestra and start working away, knowing your global variables will be
initialized and ready to go.

Pre blocks are useful if you need the enclosed ORC code to be generated
before any other global orc code. When blue goes to process the project
to generate a CSD file, it normally starts of the existing global
orchestra code from the globals manager, then appends all global orc
from instruments and soundObjects to it, one after the other. By using a
pre block, the code enclosed will be pushed up to the top of the global
orchestra of the CSD (instrument 0).

The scenario which required a solution such as this was when working
with the fluidSynth opcodes. For those set of opcodes, a fluidEngine
needs to be initialized before loading of soundfonts into that engine.
The design I had wanted was to have the instrument that did the
fluidOutput also have the engine initialization. This was a problem
because the fluidOutput needed to be placed at an instrument number
higher than the fluidNote playing instruments for the output chain to
work correctly. So here was a case of an instrument that needed to have
its instrument generate later than the instruments that it depended on,
but also have its own fluidEngine initialization happen before all the
other soundFonts load in in their own global orc statements. The
solution then was to create the once block feature and allow for
targetting global orc code to be prepended to the top of the generated
global orc code.
