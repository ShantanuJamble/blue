\section{Sound Object Freezing}\label{soundObjectFreezing}

Sound Object Freezing allows you to free up CPU-cycles by pre-rendering
soundObjects. Frozen soundObjects can work with global processing
instruments, and files are relative to the directory the project file is
in, so can be moved from computer to computer without problem. Frozen
soundObjects can be unfrozen at anytime, returning the original
soundObject and removing the frozen wave file.

To freeze a soundObject, select one or many soundObjects on the
timeline, rt-click on a selected soundObject, and then select
"Freeze/Unfreeze SoundObjects". To unfreeze, select one or many frozen
soundObjects and select the same menu option.

On the timeline, if your soundObject rendered wave is longer in duration
than the original soundObject's duration (as is the case if you have
reverb processing), the frozen soundObject's bar will graphically show
the difference in times with two different colors.

Note: As currently implemented, when blue goes to freeze soundObjects it
may appear to be frozen, but messages will continue to appear in the
console showing that csound is rendering the frozen soundObjects. Future
versions will be more polished.

\begin{enumerate}
\def\labelenumi{\arabic{enumi}.}
\item
  An soundObject is selected
\item
  Using the same project settings (all of the instruments, tables,global
  orc/sco, etc.) but not scoreTimeline generated sco, blue generates the
  sco for the selected soundObject and produce a temporary .csd file
\item
  blue runs csound with "csound -Wdo freezex.wav tempfile.csd" where the
  x in freezex.wav is an integer, counting up. This wav file is
  generated in the same directory that the projectFile is located.
\item
  blue replaces the soundObject in the timeline with a
  FrozenSoundObject. The FrozenSoundObject keeps a copy of the original
  soundObject (for unfreezing), as well as shows the name of the frozen
  wav file, the original soundObject's duration, and the frozen wav
  file's duration (not necessarily the same, as is the case if using
  global reverb, for example).
\item
  When you do a render of the entire piece now, the frozen sound object
  generates a very simple wav playing csound instrument that will play
  the rendered wav file as-is. The instrument looks something like:

\begin{verbatim}
aout1, aout2    diskin    p4         
                outs      aout1, aout2 
          
\end{verbatim}

  and the FrozenSoundObject only generates a single note that has the
  start-time, the duration of the frozen wav file, and the name of the
  file. This will end up playing the soundFile exactly as if the SCO for
  the original soundObject was generated. This also bypasses any routing
  to global sound processing, as if you had any of these effects
  originally, the would be generated as part of the frozen file.
\end{enumerate}

\begin{itemize}
\item
  You can select multiple soundObjects and batch freeze and unfreeze
  -the generated wav file may be longer than the original soundObject,
  due to global processing instruments (like reverb, echo, etc.) This is
  taken into account.
\item
  The freezing system does *not* work for all graph toplogies. If you're
  using soundObjects with instruments used as control signals, this
  won't work unless the notes for the instruments they are controlling
  are alsoin the same soundObject. I.e. I have one soundObject that has
  only notes that affect global variables, while I have one instrument
  thatuses those global variables. This could work though if you
  repackage the set of soundObjects into a polyObject. Probably best to
  generalize as:

  \begin{itemize}
  \item
    Your soundObject must be self-contained
  \item
    All sound output from instruments go directly out or piped through
    always-on instruments, that most likely should take advantage of the
    \textless{}total\_dur\textgreater{} variable, as well as the new
    \textless{}processing\_start\textgreater{} variable (more about this
    when I release, but together with freezing, this lets you set the
    start time of always-on instruments to the first time where
    non-frozen soundObjects occur, so if the first half of your piece is
    frozen and you're unfrozen stuff is in the second half, you don't
    need always on instruments to be turned on until the second half as
    the first half is routed to outs
  \end{itemize}
\item
  This system is tested with 2-channel pieces. I'm not sure if this will
  work with higher number of channels, but I don't see why it wouldn't.
\item
  Changing the number of channels on the project after a freeze may
  cause Csound errors when rendering the frozen soundObject (can be
  remedied by unfreezing and refreezing)
\item
  Frozen files are referenced relatively to the project file, so you are
  free to move your project directory around or rename it and the frozen
  files will work fine.
\end{itemize}
